\documentclass{article}
\usepackage{fontspec}
\usepackage{polyglossia}
\usepackage{pgfplots}
\usepackage{tikz}
\usepackage{unicode-math}
\usepackage{amsmath}
\usepackage{listings}

\setmainlanguage{french}

\newcommand{\R}{\mathbb{R}}

\DeclareMathOperator{\card}{card}

\title{Ocaml---IA\\Tetris par Qlearning}
\author{C.~Cousin, G.~Hondet, L.~Pineau, B.~Viry}
\date{\today}


\begin{document}
\maketitle
\tableofcontents

\section*{Introduction}

\section*{Notations}
Seront not\'es dans tout l'ouvrage:
\begin{itemize}
  \item \(n_t\) le nombre de tetrominos par jeu,
  \item \(\mathcal{S}\) l'ensemble des \'etats,
  \item \(b_w, b_h\) la largeur et la hauteur du plateau.
\end{itemize}
Sauf mention explicite, les applications utiliseront les valeurs \(n_t=10,000,
b_w = 6\).

\section{Tetris}

\subsection{Le jeu et sa simplification}
Tetris met le joueur au défi de réaliser des lignes complètes en déplaçant des
pièces de formes différentes, les tétrominos, qui défilent depuis le haut
jusqu'au bas de l'écran. Les lignes complétées disparaissent tout en rapportant
des points et le joueur peut de nouveau remplir les cases libérées. Le jeu
n'a pas de fin: le joueur perd la partie lorsqu'un tétrimino reste bloqué en
haut.

Dans notre version simplifiée, il n'y a pas de limite de hauteur, le jeu se
termine apres avoir fourni 10,000 tetrominos à placer. De plus notre
tetris diffère par une grille plus étroite (6 colonnes disponibles), moins de
tetrominos différents (seulement 5) et des tetrominos plus petit (taille
inférieur à 2).

Un tetromino est plac\'e par le choix d'une colonne et d'une orientation.

\subsection{L'impl\'ementation}

\subsubsection{Representation}

\paragraph{Tetrominos}
Les tetrominos sont tableaux longueur 4. L'unidimensionnalit\'e permet
d'impl\'ementer les rotations comme des transformations d'indices, \'evitant la
cr\'eation ou la modification de structures.

IMAGE NED EED\\
Les tetrominos sont tous de même dimension par souci d'homogeneite.

\paragraph{Plateau \texttt{Board}}
Le plateau est repr\'esent\'e par une matrice de dimension
\(2\cdot n_t\cdot 6\) d'entiers dans lequel on inscrit les tetrominos. Les
manipulations du plateau se font en place.


\paragraph{Actions}
Composée d'une rotation et d'une translation. Elles permettent d'agir sur les
tetrominos. Les rotations sont representées par les points cardinaux
(North, South, East, West) et les translations par un entier entre 0 et 4
correspondant à l'indice du coin superieur gauche du tetromino.

POINT DE REF

\subsection{D\'eroulement d'une partie}
Lors d'une partie le joueur doit poser les pièces qui lui sont proposées sur
le plateau de jeu. Il est donc nécessaire d'utiliser une fonction play qui
permet de placer une piece sur le plateau. La pièce ``tombe'' donc dans la
colonne sélectionnée tant qu'elle ne rencontre pas d'obstacle (fond du plateau
ou un autre tétromino). Pour cela, il y a besoin d'une fonction de test de
collision, qui renvoie si l'emplacement testé peut accueillir le tetromino.
S'il y a collision, on sélectionne le dernier emplacement libre et on place la
pièce.

Lorsqu'une ligne est entièrement remplie, elle est supprimée du tableau, mais
la taille totale du plateau reste la même.

\subsection{Drawbacks}
\subsubsection{fct collide}
L'implementation actuelle oblige l'accessibilit\'e de la position finale depuis
le haut du plateau car la pi\`ece ne peut pas \^etre gliss\'ee au dernier moment
sous une autre.
\subsubsection{tetromino}
Les tetrominos sont tous de même dimension par souci d'homogeneite et cela
peut entrainer un probleme de ``tetromino flottant''. Ce problème est traité par
un choix d'action spécifique à chaque pièce (cf partie x).
\section{Q learning}

\subsection{Representation des etats}
\subsubsection{Composition}
Intuitivement, un etat correspond a la piece donnee a jouer et la disposition
des pieces sur le plateau. En notant \(n_t\) le nombre de tetrominos et \(w\) la
largeur du plateau, on obtient \(5 \cdot 2^{wn_t}\) etats possibles.

Pour reduire le nombre d'etats, on ne considere que les deux plus hautes lignes
contenant au moins un tetromino. On obtient \(5\cdot 2^{w+1}\) etats possibles,
soit \(20,480\) pour une largeur de 6.

\subsubsection{Codage}
Comme chaque etat correspond \textit{in fine} a une ligne de la matrice, il est
necessaire d'avoir une bijection entre l'ensemble des etats et les entiers. La
bijection genere deux representations entieres, une pour les deux lignes de
l'etat et l'autre pour la piece et les combine en un etat entier final,
\[
  \texttt{get\_state}\colon [0,4]\times [0, 2^{12} - 1] \to [0, 5\cdot 2^{12}].
\]


\subsection{Algorithmes}

\subsection{Param\`etres}
\subsubsection{Pr\'esentation}
L'apprentissage est parametre par les trois variables,
\begin{itemize}
  \item \(\alpha \in [0, 1]\) le taux d'apprentissage,
  \item \(\epsilon \in [0, 1]\) la frequence de coups aleatoires effectues,
  \item \(\gamma \in [0, 1]\) la vision de l'agent.
\end{itemize}

Pour assurer la convergence de la matrice vers la matrice optimale,
le taux d'apprentissage evolue au cours des entrainements, est utilisee une
suite \( (\alpha_k)_k \). D'apres~\cite{watkins92}, la suite \(
(\alpha_k)_k \) doit verifier \( \sum_{k=0}^\infty \alpha_k = \infty \) et \(
\sum_{k=0}^\infty \alpha_k^2 \in \R \). La suite choisie est donc, pour tout
\( k \in \mathbb{N} \) et avec \( C \in \R \)
\[
  \alpha_k = \frac{1}{1 + Ck}.
\]
Le taux d'apprentissage reste manipulable via le parametre \( C \).


Le parametre \(\epsilon\) pourra \'egalement varier au cours des jeux effectues.
En effet, intuitivement, un \(\epsilon\) grand permet une exporation rapide de
l'ensemble des etats possibles mais devient nuisible lorsque la matrice est bien
entrainee.

\subsubsection{Valeurs}
Par defaut les valeurs utilisees


\section{Difficultes rencontrees}

\section{R\'esultats du projet}

\section*{Conclusion}

\appendix
\section{Deep Q learning}

\begin{thebibliography}{9}
    \bibitem{watkins92}
    Christopher~J.C.H.~Watkins, Peter~Dayan
    \textit{Q-Learning, Machine Learning}
    1992.
    \bibitem{deepmind}
    Voloddymyr~Mnih, Koray Kavukcuoghlu, David Silver, Alex Graves, Ioannis
    Antonoglou, Daan Wierstra, Martin Riedmiller,
    \textit{Playing Atari with Deep Reinforcement Learning}
\end{thebibliography}

\end{document}
