\documentclass{article}
\usepackage{fontspec}
\usepackage{polyglossia}
\usepackage{pgfplots}
\usepackage{tikz}
\usepackage{unicode-math}
\usepackage{amsmath}

\setmainlanguage{french}

\DeclareMathOperator{\card}{card}

\title{Ocaml---IA\\Tetris par Qlearning}
\author{C.~Cousin, G.~Hondet, L.~Pineau, B.~Viry}
\date{\today}


\begin{document}
\maketitle
\tableofcontents


\section*{Introduction}

\section{Tetris}

\subsection{Le jeu et sa simplification}
Tetris met le joueur au défi de réaliser des lignes complètes en déplaçant des
pièces de formes différentes, les tétrominos, qui défilent depuis le haut
jusqu'au bas de l'écran. Les lignes complétées disparaissent tout en rapportant
des points et le joueur peut de nouveau remplir les cases libérées. Le jeu
n'a pas de fin: le joueur perd la partie lorsqu'un tétrimino reste bloqué en
haut.

Dans notre version simplifiée, il n'y a pas de limite de hauteur, le jeu se
termine après que je jeu ai donné 10,000 tetrominos à placer. De plus notre
tetris diffère par une grille plus étroite (6 colonnes disponibles), moins de
tetrominos différents (seulement 5) et des tetrominos plus petit (taille
inférieur à 2).

Pour placer le tetrominos le joueur peut choisir la colone et l'orientation de
la pièce. Contrairement au tetris classique, les pièces ne défilent pas du haut
de l'écran mais sont directement placé en jeu et l'on a plus d'emprise sur elle
une fois l'action choisie.

\subsection{L'impl\'ementation}

\subsection{Representation}

\subsubsection{Plateau \textit{board}}

\subsubsection{Tetrominos}

\section{Q learning}

\subsection{Representation des etats}
\subsubsection{Composition}
Intuitivement, un etat correspond a la piece donnee a jouer et la disposition
des pieces sur le plateau. En notant \(n_t\) le nombre de tetrominos et \(w\) la
largeur du plateau, on obtient \(5 \cdot 2^{wn_t}\) etats possibles.

Pour reduire le nombre d'etats, on ne considere que les deux plus hautes lignes
contenant au moins un tetromino. On obtient \(5\cdot 2^{w+1}\) etats possibles,
soit \(20,480\) pour une largeur de 6.

\subsubsection{Codage}
Comme chaque etat correspond \textit{in fine} a une ligne de la matrice, il est
necessaire d'avoir une bijection entre l'ensemble des etats et les entiers. La
bijection genere deux representations entieres, une pour les deux lignes de
l'etat et l'autre pour la piece et les combine en un etat entier final,
\[
  \texttt{get\_state}\colon [0,4]\times [0, 2^{12} - 1] \to [0, 5\cdot 2^{12}].
\]

\section{Difficultes rencontrees}

\section{R\'esultats du projet}

\section*{Conclusion}

\appendix
\section{Deep Q learning}

\end{document}
