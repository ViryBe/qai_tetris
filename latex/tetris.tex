\documentclass{article}
\usepackage{listings}       % code
\usepackage{fourier}        % danger
\usepackage{graphicx}

\title{Super projet de la mort qui tue}
\author{...}

\begin{document}
\maketitle
\lstset{language=Caml}

\section{Notations}
\begin{description}
  \item[tetromino] une piece du tetris manipulable par l'agent;
  \item[board] le plateau de jeu, le receptacle des tetrominos;
  \item[action] le placement d'un tetromino sur le plateau;
  \item[game] une partie de tetris, soit la succession de 10,000 actions.
\end{description}

\section{Tetris}

\subsection{Types}
\begin{lstlisting}[frame=L]
(* Type plateau (simple alias)  *)
type plateau = int array array
\end{lstlisting}

\begin{lstlisting}[frame=L]
(* Type piece (simple alias)  *)
type piece = int array array
\end{lstlisting}

\begin{lstlisting}[frame=L]
(* Type mouvement (rotation + translation )*)
type mouvement =
  | Nord
  | Sud
  | Est
  | Ouset
  | Placement of int (* sur quelle ligne on place la piece *)
\end{lstlisting}

\subsection{Fonctions}


\begin{lstlisting}[frame=L]
(* Fonction qui place une nouvelle piece sur le plateau
 * et renvoie le plateau resultant *)
val next_step : plateau -> piece -> mouvement -> plateau
\end{lstlisting}

\begin{lstlisting}[frame=L]
(* Donne une piece random *)
val random_piece : () -> piece
\end{lstlisting}

\subsubsection{Suggestions d'implementation}

Details d'impelementations.

\section{Joueur, AI}

Grand debat sur la matrice Q, les états...

\subsection{Types}

\subsection{Fonctions}

\danger{} le type de \( Q \) est flottant a cause de \( \epsilon \), \( \alpha \) et \( \gamma \).

\noindent
\danger{} les parametres des fonctions sont a changer en fonction de notre
choix de representation des etats.

\begin{lstlisting}[frame=L]
(* fonction qui calcul le learning rate a l'etape k
 * Parametres k : int
 *)
val alpha : int -> float
\end{lstlisting}


\begin{lstlisting}[frame=L]
(* fonction qui determine le prochain coup a jouer
 * Parametres :
 * 1 : matrice Q
 * 2 : epsilon (combien de coup aléatoire)
 * 3 : plateau en cours
 * 4 : piece a placer
 * 5 : liste des actions possibles
 * Retour :
 * le prochain coup a jouer
 *)

val play : float array array -> float -> plateau -> piece ->
	mouvement array -> mouvement
\end{lstlisting}

\begin{lstlisting}[frame=L]
(* fonction d'entrainement du modele
 * Paramètres :
 * 1 : matrice Q
 * 2 : epsilon (combien de coup aléatoire)
 * 3 : gamma ('memoire' du modele)
 * 4 : alpha (learining rate)
 * 5 : n nombres de jeux d'entrainement
 * Retour :
 * matrice Q apres entrainement
 *)

val training : float array array -> float -> float ->
	(fun int -> float) -> int -> float array array
\end{lstlisting}

\subsection{Suggestions d'implementation}

\begin{lstlisting}[frame=L]
(* fonction qui renvoie le reward a partir d'un plateau *)
val evaluator : plateau -> float
\end{lstlisting}

\begin{lstlisting}
(* calcul du prochain etat a y reflechir *)
\end{lstlisting}

\section{Main}

Principalement initialisation du jeu (plateau) et des constantes (pieces et
mouvement qui sont predefini).


\subsection{Options:}
\begin{itemize}
	\item n : nombre de coup a jouer
	\item epsilon : proba des coups aleatoires
	\item gamma : vison de l'agent
	\item matrice Q (sous forme de nom de fichier)
\end{itemize}
% \begin{center}
%     \includegraphics[height=4cm]{images.duckduckgo.com.jpeg}
% \end{center}



\end{document}
